\documentclass[12pt,a4paper]{article}
\usepackage[T1]{fontenc}
\usepackage[utf8]{inputenc}
\usepackage{dsfont} 
\usepackage[polish]{babel}
\usepackage{amsmath}
\usepackage{graphicx}
\usepackage[top=1in, bottom=1.5in, left=1.25in, right=1.25in]{geometry}
\usepackage{hyperref}

\usepackage{subfig}
\usepackage{multirow}
\usepackage{multicol}
\graphicspath{{Images/}}
\usepackage{xcolor,colortbl}
\usepackage{float}

\newcommand \comment[1]{\textbf{\textcolor{red}{#1}}}

%\usepackage{float}
\usepackage{fancyhdr} % Required for custom headers
\usepackage{lastpage} % Required to determine the last page for the footer
\usepackage{extramarks} % Required for headers and footers
\usepackage{indentfirst}
\usepackage{placeins}
\usepackage{scalefnt}
\usepackage{xcolor,listings}
\usepackage{textcomp}
\usepackage{color}
\usepackage{verbatim}
\usepackage{framed}

\definecolor{codegreen}{rgb}{0,0.6,0}
\definecolor{codegray}{rgb}{0.5,0.5,0.5}
\definecolor{codepurple}{HTML}{C42043}
\definecolor{backcolour}{HTML}{F2F2F2}
\definecolor{bookColor}{cmyk}{0,0,0,0.90}  
\color{bookColor}

\lstset{upquote=true}

\lstdefinestyle{mystyle}{
	backgroundcolor=\color{backcolour},   
	commentstyle=\color{codegreen},
	keywordstyle=\color{codepurple},
	numberstyle=\numberstyle,
	stringstyle=\color{codepurple},
	basicstyle=\footnotesize\ttfamily,
	breakatwhitespace=false,
	breaklines=true,
	captionpos=b,
	keepspaces=true,
	numbers=left,
	numbersep=10pt,
	showspaces=false,
	showstringspaces=false,
	showtabs=false,
}
\lstset{style=mystyle}

\newcommand\numberstyle[1]{%
	\footnotesize
	\color{codegray}%
	\ttfamily
	\ifnum#1<10 0\fi#1 |%
}

\definecolor{shadecolor}{HTML}{F2F2F2}

\newenvironment{sqltable}%
{\snugshade\verbatim}%
{\endverbatim\endsnugshade}

% Margins
\addtolength{\footskip}{0cm}
\addtolength{\textwidth}{1.4cm}
\addtolength{\oddsidemargin}{-.7cm}

\addtolength{\textheight}{1.6cm}
%\addtolength{\topmargin}{-2cm}

% paragrafo
\addtolength{\parskip}{.2cm}

% Set up the header and footer
\pagestyle{fancy}
\rhead{\hmwkAuthorName} % Top left header
\lhead{\hmwkClass: \hmwkTitle} % Top center header
\rhead{\firstxmark} % Top right header
\lfoot{Mateusz Domino} % Bottom left footer
\cfoot{} % Bottom center footer
\rfoot{} % Bottom right footer
\renewcommand{\headrulewidth}{1pt}
\renewcommand{\footrulewidth}{1pt}

    
\newcommand{\hmwkTitle}{Konfiguracja klastra Kubernetes dla aplikacji Games-Store} % Tytuł projektu
\newcommand{\hmwkDueDate}{\today} % Data 
\newcommand{\hmwkClass}{Technologie chmurowe} % Nazwa przedmiotu
\newcommand{\hmwkAuthorName}{Mateusz Domino} % Imię i nazwisko

% trabalho 
\begin{document}
% capa
\begin{titlepage}
    \vfill
	\begin{center}
	\hspace*{-1cm}
	\vspace*{0.5cm}
    \includegraphics[scale=0.55]{images/loga.png}\\
	\textbf{Uniwersytet Gdański \\ [0.05cm]Wydział Matematyki, Fizyki i Informatyki \\ [0.05cm] Instytut Informatyki}

	\vspace{0.6cm}
	\vspace{4cm}
	{\huge \textbf{\hmwkTitle}}\vspace{8mm}
	
	{\large \textbf{\hmwkAuthorName}}\\[3cm]
	
		\hspace{.45\textwidth} %posiciona a minipage
	   \begin{minipage}{.5\textwidth}
	   Projekt z przedmiotu technologie chmurowe na kierunku informatyka profil praktyczny na Uniwersytecie Gdańskim.\\[0.1cm]
	  \end{minipage}
	  \vfill
	%\vspace{2cm}
	
	\textbf{Gdańsk}

	\textbf{26 czerwca 2024}
	\end{center}
	
\end{titlepage}

\newpage
\setcounter{secnumdepth}{5}
\tableofcontents
\newpage

\section{Opis projektu}
\label{sec:Project}

Firma Games-Store Corporation, specjalizująca się w handlu grami, zleciła stworzenie nowej aplikacji webowej do sprzedaży gier różnych kategorii online. Aplikacja została stworzona przez pododział firmy. Serwis ma umożliwiać logowanie i rejestrację użytkowników, zakup oraz zwroty gier, oraz obsługiwać różne metody płatności. Również ma być dostępny panel administratora do dodawania nowych gier, dostępny tylko dla pracowników.

\subsection{Opis architektury}
\label{sec:introduction}
Architektura została zaimplementowana w środowisku Kubernetes, co umożliwia w razie potrzeby łatwe skalowanie i wdrożania nowych zmian za pomocą obrazów Docker. Węzły  klastra Kubernetes: 
\begin{itemize}
    \item Express-server - API do przetwarzania zapytań z frontendu oraz zwracania danych z bazy danych.
    \item Frontend - Aplikacja frontend napisana w frameworku Next.js.
    \item Mongo-db - Baza danych MongoDB.
    \item Nginx-proxy - Serwer Reverse-Proxy Nginx, przekazuje zapytania do aplikacji.
   
\end{itemize}

\subsection{Opis infrastruktury}
\label{sec:Users}

\subsubsection{Środowisko uruchomieniowe}
Aplikacja wykorzystuje środowisko Docker w następującej wersji silnika:
\begin{itemize}
    \item version 24.0.6
    \item build ed223b
\end{itemize}

Wersja wykorzystywanego Kubernetes: 
\begin{itemize}
    \item Client Version: v1.27.2
    \item Kustomize Version: v5.0.1
    \item Server Version: v1.27.2
\end{itemize}

\subsubsection{Zasoby obliczeniowe}
Każda replika węzła w klastrze ma zdefiniowane minimalne i maksymalne zasoby obliczeniowe. Repliki są automatycznie tworzone i skalowane w zależności od aktualnego obciążenia, wykorzystując optymalnie zasoby.

\subsubsection{Pamięć masowa}
Dla bazy danych MongoDB, jest przydzielony wolumen PersistentValueClaim, o wielkości 5Gi na dane.


\subsection{Opis komponentów aplikacji}
\label{sec:FunctionalConditions}
Aplikacja składa się się z 4 komponentów, każdy został zdefiniowany w oddzielnym pliku manifest dla Kubernetesa jako osobne pody:

\subsubsection{Baza danych MongoDB}
Służy do przechowywania informacji o użytkownikach, grach, powiadomieniach, wiadomościach pomocy technicznej, transakcjach oraz zwrotów gier. Do tego użyto bazę danych MongoDB z odpowiednimi kolekcjami dla danych.

\subsubsection{API Express node.js}
Api zostało stworzone w node.js używając framework Express. Służy do obsługi zapytań wysyłanych przez Frontend, obsługuje przetwarzanie transakcji, użytkowników, przekazuje informacje z bazy danych oraz waliduje tokeny logowania. Również zawiera operacje dostępne tylko dla pracowników sklepu, między innymi: dodawanie nowych gier, przetwarzanie zwrotów gier, odpowiadanie na wiadomości pomocy technicznej oraz zarządzanie użytkownikami. Do tego jest wykorzystywany zmodyfikowany obraz node zawierający aplikację. 

\subsubsection{Frontend next.js}
Frontend został zaimplemenowany w frameworku Next.js, operatym na React. Użytkownik za jego pomocą może wykonywać wszystkie operacje w aplikacji, takie jak logowanie, rejestracja, zakup gier i ich zwracanie, wysyłanie wiadomości do pomocy, sprawdzanie powiadomień, zmiana informacji profilu takich jak na przykład nazwa użytkownika i hasło. Zawiera również panel administratora. Komunikacja z API polega na wysyłaniu zapytań HTTP. Został w tym celu wykorzystany zmodyfikowany obraz node zawierający skompilowaną aplikację.

\subsubsection{Serwer Reverse Proxy Nginx}
Jego zadaniem jest przekierowywanie zapytań do poszczególnych komponentów, działa na obrazie nginx.

\newpage

\subsection{Konfiguracja i zarządzanie}
\label{sec:NonFunctionalConditions}

Konfiguracja aplikacji jest zdefiniowana w plikach manifest Kubernetes, które określają serwisy i deploymenty klastra. Zawarte są w nich poprzez zmienne środowiskowe, między innymi: secret dla szyfrowania/odszyfrowania hasła oraz adres bazy danych. Zawiera deployment 4 głównych komponentów, poprzez obrazy Docker. Do komunikacji służy serwer Nginx ustawiony jako LoadBalancer, do przekazywania zapytań zarówno do Api, jak i do frontendu.

\subsection{Zarządzanie błędami}
\label{sec:ERD} 

Api jest przystosowane do wysłania odpowiednich komunikatów w razie, błędnych zapytań wraz z kodami błędów.

W przypadku wystąpienia błądu w jednym z podów, Kubernetes automatycznie go usuwa, po czym tworzy nowy, identyczny. W ten sposób w środowisku produkcyjnym, zapewnione jest ciągłe działanie aplikacji.

\subsection{Skalowalność}
\label{sec:ExamplesSection}
Skalowanie aplikacji jest realizowane przez Horizontal Pod Autoscaler (HPA). Przy obecnej konfiguracji monitorowane jest Api, frontend oraz serwer Nginx. W razie zwiększonego obciązania któregoś z tych elementów, HPA automatycznie zwiększy ilość jego replik, a w razie spadku proporcjonalnie zmniejszy. Aktualnie HPA skonfigurowane jest na minimalnie po 3 repliki dla każdego z wymienionego elementu, a maksymalnie na 10. Średnie użycie HPA jest ustawione na 50 dla CPU i 70 dla pamięci RAM.

\newpage

\subsection{Wymagania dotyczące zasobów}
\label{sec:ExampleTables}

Wymagania zasobów dla każdego komponenetu aplikacji, z przyjętym oczekiwanym czasem odpowiedzi poniżej 200 ms.

\subsubsection{Baza danych}
\begin{itemize}
    \item Wymagania minimalne
    \begin{itemize}
        \item CPU: 1
        \item RAM: 2Gi
    \end{itemize}
    
    \item Wymagania maksymalne
    \begin{itemize}
        \item CPU: 3
        \item RAM: 4Gi
    \end{itemize}
\end{itemize}

\subsubsection{Api}
\begin{itemize}
    \item Wymagania minimalne
    \begin{itemize}
        \item CPU: 0.5
        \item RAM: 256Mi
    \end{itemize}
    
    \item Wymagania maksymalne
    \begin{itemize}
        \item CPU: 1
        \item RAM: 512Mi
    \end{itemize}
\end{itemize}

\subsubsection{Frontend}
\begin{itemize}
    \item Wymagania minimalne
    \begin{itemize}
        \item CPU: 0.1
        \item RAM: 128Mi
    \end{itemize}
    
    \item Wymagania maksymalne
    \begin{itemize}
        \item CPU: 0.5
        \item RAM: 512Mi
    \end{itemize}
\end{itemize}

\newpage

\subsubsection{Serwer Reverse Proxy}
\begin{itemize}
    \item Wymagania minimalne
    \begin{itemize}
        \item CPU: 0.1
        \item RAM: 64Mi
    \end{itemize}
    
    \item Wymagania maksymalne
    \begin{itemize}
        \item CPU: 0.5
        \item RAM: 256Mi
    \end{itemize}
\end{itemize}

\subsection{Architektura sieciowa}
\label{sec:ExampleResults}

\subsubsection{Baza danych}
Serwis typu NodePort, nasłuchujący na porcie 30332:27017. Ten serwis pozwala do dostęp do bazy z zewnątrz klastra na porcie 30332 na każdym węźle.

\subsubsection{Api}
Serwis typu ClusterIP, zapewnia dostęp do Api Express. Nasłuchuje na porcie 8000:8000. Nie można się dostać do niego z zewnątrz klastra.

\subsubsection{Frontend}
Serwis typu ClusterIP, zapewnia dostęp do aplikacji frontend. Nasłuchuje na porcie 3000:3000. Nie można się dostać do niego z zewnątrz klastra.

\subsubsection{Serwer Reverse Proxy}

Serwer Nginx to jedyny serwis typu LoadBalancer, przez co jako jedyny ma przydzielony zewnętrzny adres IP. Serwer nasłuchuje na porcie 80 i w zaleznosci od  ścieżki URL, przekazuje zapytania do frontendu bądź Api.

\newpage

\bibliographystyle{amsplain}
\bibliography{references.bib}
\nocite{*}

\end{document}